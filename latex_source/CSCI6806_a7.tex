%%
%% This is file `sample-sigconf.tex',
%% generated with the docstrip utility.
%%
%% The original source files were:
%%
%% samples.dtx  (with options: `all,proceedings,bibtex,sigconf')
%% 
%% IMPORTANT NOTICE:
%% 
%% For the copyright see the source file.
%% 
%% Any modified versions of this file must be renamed
%% with new filenames distinct from sample-sigconf.tex.
%% 
%% For distribution of the original source see the terms
%% for copying and modification in the file samples.dtx.
%% 
%% This generated file may be distributed as long as the
%% original source files, as listed above, are part of the
%% same distribution. (The sources need not necessarily be
%% in the same archive or directory.)
%%
%%
%% Commands for TeXCount
%TC:macro \cite [option:text,text]
%TC:macro \citep [option:text,text]
%TC:macro \citet [option:text,text]
%TC:envir table 0 1
%TC:envir table* 0 1
%TC:envir tabular [ignore] word
%TC:envir displaymath 0 word
%TC:envir math 0 word
%TC:envir comment 0 0
%%
%% The first command in your LaTeX source must be the \documentclass
%% command.
%%
%% For submission and review of your manuscript please change the
%% command to \documentclass[manuscript, screen, review]{acmart}.
%%
%% When submitting camera ready or to TAPS, please change the command
%% to \documentclass[sigconf]{acmart} or whichever template is required
%% for your publication.
%%
%%
\documentclass[sigconf]{acmart}
%%
%% \BibTeX command to typeset BibTeX logo in the docs
\AtBeginDocument{%
  \providecommand\BibTeX{{%
    Bib\TeX}}}

%% Rights management information.  This information is sent to you
%% when you complete the rights form.  These commands have SAMPLE
%% values in them; it is your responsibility as an author to replace
%% the commands and values with those provided to you when you
%% complete the rights form.
% \setcopyright{acmlicensed}
\setcopyright{none}
% \copyrightyear{2018}
% \acmYear{2018}
% \acmDOI{XXXXXXX.XXXXXXX}
%% These commands are for a PROCEEDINGS abstract or paper.
% \acmConference[Conference acronym 'XX]{Make sure to enter the correct
%   conference title from your rights confirmation email}{June 03--05,
%   2018}{Woodstock, NY}
%%
%%  Uncomment \acmBooktitle if the title of the proceedings is different
%%  from ``Proceedings of ...''!
%%
%%\acmBooktitle{Woodstock '18: ACM Symposium on Neural Gaze Detection,
%%  June 03--05, 2018, Woodstock, NY}
% \acmISBN{978-1-4503-XXXX-X/2018/06}
\acmConference{CSCI6806 Capstone Proj}{Sep. 2025}{Vancouver, BC, CA}

\settopmatter{printacmref=false} % removes the footnote below the first column
\renewcommand\footnotetextcopyrightpermission[1]{} % removes conference info footnote

%%
%% Submission ID.
%% Use this when submitting an article to a sponsored event. You'll
%% receive a unique submission ID from the organizers
%% of the event, and this ID should be used as the parameter to this command.
%%\acmSubmissionID{123-A56-BU3}

%%
%% For managing citations, it is recommended to use bibliography
%% files in BibTeX format.
%%
%% You can then either use BibTeX with the ACM-Reference-Format style,
%% or BibLaTeX with the acmnumeric or acmauthoryear sytles, that include
%% support for advanced citation of software artefact from the
%% biblatex-software package, also separately available on CTAN.
%%
%% Look at the sample-*-biblatex.tex files for templates showcasing
%% the biblatex styles.
%%

%%
%% The majority of ACM publications use numbered citations and
%% references.  The command \citestyle{authoryear} switches to the
%% "author year" style.
%%
%% If you are preparing content for an event
%% sponsored by ACM SIGGRAPH, you must use the "author year" style of
%% citations and references.
%% Uncommenting
%% the next command will enable that style.
%%\citestyle{acmauthoryear}

% fix figure
\usepackage{float}
\usepackage{makecell}

\usepackage[inkscapelatex=false]{svg}
\svgsetup{inkscapelatex=false}


%%
%% end of the preamble, start of the body of the document source.
\begin{document}

%%
%% The "title" command has an optional parameter,
%% allowing the author to define a "short title" to be used in page headers.
\title{Group 1: Artifact Evaluation}

%%
%% The "author" command and its associated commands are used to define
%% the authors and their affiliations.
%% Of note is the shared affiliation of the first two authors, and the
%% "authornote" and "authornotemark" commands
%% used to denote shared contribution to the research.

\vfill
\author{Anna Gorislavets}
\affiliation{%
  \institution{Fairleigh Dickinson University}
  \city{Vancouver}
  \country{Canada}
}
\email{a.gorislavets@student.fdu.edu}

\author{Bikash Shyangtang}
\affiliation{%
  \institution{Fairleigh Dickinson University}
  \city{Vancouver}
  \country{Canada}
}
\email{b.shyangtang@student.fdu.edu}

\author{Hao Chen}
\affiliation{%
  \institution{Fairleigh Dickinson University}
  \city{Vancouver}
  \country{Canada}
}
\email{h.chen4@student.fdu.edu}

\author{Maoting Li}
\affiliation{%
  \institution{Fairleigh Dickinson University}
  \city{Vancouver}
  \country{Canada}
}
\email{m.li3@student.fdu.edu}

\author{Salinrat Thanathapsakun}
\affiliation{%
  \institution{Fairleigh Dickinson University}
  \city{Vancouver}
  \country{Canada}
}
\email{s.thanathapsakun@student.fdu.edu}
\vfill

%%
%% By default, the full list of authors will be used in the page
%% headers. Often, this list is too long, and will overlap
%% other information printed in the page headers. This command allows
%% the author to define a more concise list
%% of authors' names for this purpose.
% \renewcommand{\shortauthors}{Trovato et al.}

%%
%% The abstract is a short summary of the work to be presented in the
%% article.
% \begin{abstract}
%   A clear and well-documented \LaTeX\ document is presented as an
%   article formatted for publication by ACM in a conference proceedings
%   or journal publication. Based on the ``acmart'' document class, this
%   article presents and explains many of the common variations, as well
%   as many of the formatting elements an author may use in the
%   preparation of the documentation of their work.
% \end{abstract}

%%
%% The code below is generated by the tool at http://dl.acm.org/ccs.cfm.
%% Please copy and paste the code instead of the example below.
%%
% \begin{CCSXML}
% <ccs2012>
%  <concept>
%   <concept_id>00000000.0000000.0000000</concept_id>
%   <concept_desc>Do Not Use This Code, Generate the Correct Terms for Your Paper</concept_desc>
%   <concept_significance>500</concept_significance>
%  </concept>
%  <concept>
%   <concept_id>00000000.00000000.00000000</concept_id>
%   <concept_desc>Do Not Use This Code, Generate the Correct Terms for Your Paper</concept_desc>
%   <concept_significance>300</concept_significance>
%  </concept>
%  <concept>
%   <concept_id>00000000.00000000.00000000</concept_id>
%   <concept_desc>Do Not Use This Code, Generate the Correct Terms for Your Paper</concept_desc>
%   <concept_significance>100</concept_significance>
%  </concept>
%  <concept>
%   <concept_id>00000000.00000000.00000000</concept_id>
%   <concept_desc>Do Not Use This Code, Generate the Correct Terms for Your Paper</concept_desc>
%   <concept_significance>100</concept_significance>
%  </concept>
% </ccs2012>
% \end{CCSXML}

% \ccsdesc[500]{Do Not Use This Code~Generate the Correct Terms for Your Paper}
% \ccsdesc[300]{Do Not Use This Code~Generate the Correct Terms for Your Paper}
% \ccsdesc{Do Not Use This Code~Generate the Correct Terms for Your Paper}
% \ccsdesc[100]{Do Not Use This Code~Generate the Correct Terms for Your Paper}

%%
%% Keywords. The author(s) should pick words that accurately describe
%% the work being presented. Separate the keywords with commas.
% \keywords{Flash Cache, HDD throughput bottleneck, Disk-head Time (DT)}
%% A "teaser" image appears between the author and affiliation
%% information and the body of the document, and typically spans the
%% page.
% \begin{teaserfigure}
%   \includegraphics[width=\textwidth]{sampleteaser}
%   \caption{Seattle Mariners at Spring Training, 2010.}
%   \Description{Enjoying the baseball game from the third-base
%   seats. Ichiro Suzuki preparing to bat.}
%   \label{fig:teaser}
% \end{teaserfigure}

% \received{20 February 2007}
% \received[revised]{12 March 2009}
% \received[accepted]{5 June 2009}

%%
%% This command processes the author and affiliation and title
%% information and builds the first part of the formatted document.
\maketitle

\onecolumn

\section*{Github Repo: https://github.com/JRYOO-FDU-CAPSTONE/csci-6806-fa-2025-6806\_fa2025\_group1}

\section{Repository Initialization}

\begin{figure}[h]
  \centering
  \includegraphics[width=1\textwidth]{a7_diagrams/First-Commit.png}
  \caption{First commit} %Peak DT rises as only a few blocks are promoted to the protected segment, then declines once the threshold becomes less restrictive.}
  \label{fig:1}
\end{figure}

First commit contains only Baleen source; dev branch created.

\clearpage

\section{Branching}

\begin{figure}[h]
  \centering
  \includegraphics[width=1\textwidth]{a7_diagrams/Branching.png}
  \caption{Feature branches}
  \label{fig:1}
\end{figure}

\begin{figure}[h]
  \centering
  \includegraphics[width=1\textwidth]{a7_diagrams/Graph-Branch.png}
  \caption{Feature branches with branch graph}
\end{figure}

\clearpage

\section{Main Branch Workflow}

\begin{figure}[h]
  \centering
  \includegraphics[width=0.9\textwidth]{a7_diagrams/Merge.png}
  \caption{Weekend merge to main branch from dev branch} %Peak DT rises as only a few blocks are promoted to the protected segment, then declines once the threshold becomes less restrictive.}
  \label{fig:1}
\end{figure}

\begin{figure}[h]
  \centering
  \includegraphics[width=0.8\textwidth]{a7_diagrams/Network-graph.png}
  \caption{Branches Diagram} %Peak DT rises as only a few blocks are promoted to the protected segment, then declines once the threshold becomes less restrictive.}
  \label{fig:1}
\end{figure}

\clearpage

\section{Commit Quality \& Documentation}

\begin{figure}[h]
  \centering
  \includegraphics[width=0.9\textwidth]{a7_diagrams/Commit.png}
  \caption{Commit Example 1} %Peak DT rises as only a few blocks are promoted to the protected segment, then declines once the threshold becomes less restrictive.}
  \label{fig:1}
\end{figure}

\begin{figure}[h]
  \centering
  \includegraphics[width=0.9\textwidth]{a7_diagrams/Commit2.png}
  \caption{Commit Example 2} %Peak DT rises as only a few blocks are promoted to the protected segment, then declines once the threshold becomes less restrictive.}
  \label{fig:1}
\end{figure}

\begin{figure}[h]
  \centering
  \includegraphics[width=0.84\textwidth]{a7_diagrams/D1.png}
  \caption{Branches Diagram 1} %Peak DT rises as only a few blocks are promoted to the protected segment, then declines once the threshold becomes less restrictive.}
  \label{fig:1}
\end{figure}

\begin{figure}[h]
  \centering
  \includegraphics[width=0.84\textwidth]{a7_diagrams/D2.png}
  \caption{Branches Diagram 2} %Peak DT rises as only a few blocks are promoted to the protected segment, then declines once the threshold becomes less restrictive.}
  \label{fig:1}
\end{figure}

%%
%% If your work has an appendix, this is the place to put it.
% \appendix

% \section{Supplemental Material}

% \begin{figure}[ht!]
%   \centering
%   \includesvg[width=0.40\textwidth]{a1_diagrams/CSCI6806_storage_stack.svg}
%   \caption{Storage Architecture Diagram}
%   \label{fig:2}
% \end{figure}

\end{document}
\endinput
%%
%% End of file `sample-sigconf.tex'.
